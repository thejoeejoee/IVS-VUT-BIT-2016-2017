% \input utf8-t1
\documentclass[11pt,a4paper,titlepage]{extarticle}

\usepackage[czech]{babel}
\usepackage[utf8]{inputenc}
\usepackage{fancyhdr}
\usepackage[obeyspaces]{url}
\usepackage[paper=a4paper,top=2cm,left=2cm,right=2cm,bottom=2cm,includefoot]{geometry}
\usepackage{listings}
\setcounter{secnumdepth}{0}

\lhead{VUT FIT IVS 2016/2017}
\chead{/dej/uran/dom}
\rhead{Zpráva o profilování}
\lfoot{}
\cfoot{}

\begin{document}
\pagestyle{fancy}
\section{Zpráva o provedeném profilování}
Dle zadání bylo nad matematickou knihovnou aplikace provedeno profilování.
Zdrojový kód programu počítající směrodatnou odchylku je uložen v souboru \path{src/calculator/standard_deviation.py}
vzhledem k kořenovému adresáři projektu. Tento program je rozdělen do tří funkcí a to:
\begin{description}
\item[main] Funkce pro zkonvertování vstupního toku dat na příslušné datové typy a následné volání směrodatné odchylky.
\item[mean] Funkce, která za použití matematické knihovny spočítá aritmetický průměr hodnot předaných parametrem.
\item[standard\_deviation] Funkce pro samotný výpočet směrodatné odchylky - v parametru dostane kontejner vstupních hodnot
    a za pomocí \textbf{mean} vypočítá odchylku.
\end{description}

Za pomocí utility \path{cProfile} vestavěné ve standardní instalaci jazyka Python jsme provedli měření dle zadání - a to
 10, 100 a 1000 vstupních hodnot. Jedná se o modul generující jak uživatelsky čitelná data, tak data o profilování zpracovatelná
 programově - my použijeme rovnou uživatelsky čitelná data (ta strojově zpracovatelná budou uložena ve adresáři \path{/profiling/}):
\begin{lstlisting}[language=bash]
$ (for i in {1..10}; do echo $RANDOM; done) |
    python3 -m cProfile -s cumtime standard_deviation.py

\end{lstlisting}

\end{document}
